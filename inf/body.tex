\chapter{Компютърни системи (КС) - принципи на действие. Архитектура на Джон фон Нойман. Основни компоненти на КС - функционално предназначение и характеристики. Виртуални компютърни системи.}
% Компютърни системи (КС) -- принципи на действие. Архитектура на Джон фон Нойман. Основни компоненти на КС -- функционално предназначение и характеристики. Виртуални компютърни системи.

\chapter{Операционни системи – основни функции, класификация. Потребителски интерфейс – видове. Графичен потребителски интерфейс. Файл. Файлова система. Приложни програми.}
% Операционни системи – основни функции, класификация. Потребителски интерфейс – видове. Графичен потребителски интерфейс. Файл. Файлова система. Приложни програми.

\chapter{Компютърни мрежи. Тополoгии. Модели (OSI, TCP/IP). Протоколи (IP, TCP, UDP, DHCP, SMTP, DNS, FTP, HTTP). Адресиране. Услуги.}
% Компютърни мрежи. Тополoгии. Модели (OSI, TCP/IP). Протоколи (IP, TCP, UDP, DHCP, SMTP, DNS, FTP, HTTP). Адресиране. Услуги.

\chapter{Процедурно програмиране. Основни концепции и средства на езика за реализацията им. Основни типове данни. Контейнери. Основни управляващи конструкции. Функции. Рекурсия.}
% Процедурно програмиране. Основни концепции и средства на езика за реализацията им. Основни типове данни. Контейнери. Основни управляващи конструкции. Функции. Рекурсия.

\chapter{Обектно-ориентирано програмиране. Класове и обекти. Основни принципи в обектно-ориентираното програмиране. Атрибути и методи. Скриване на данни и интерфейси на класове. Специални атрибути и методи. Примери.}
% Обектно-ориентирано програмиране. Класове и обекти. Основни принципи в обектно-ориентираното програмиране. Атрибути и методи. Скриване на данни и интерфейси на класове. Специални атрибути и методи. Примери.

\chapter{Наследяване и полиморфизъм. Абстракция и капсулиране. Реализация на основните принципи на ООП. Обработка на изключенията. Приложения}
% Наследяване и полиморфизъм. Абстракция и капсулиране. Реализация на основните принципи на ООП. Обработка на изключенията. Приложения

\chapter{Файлове и потоци от данни. Видове файлове и основни операции с файлове. Четене и форматирано писане на текстов файл с последователен достъп. Сериализиране и десериализиране на обекти. Обработка на данни от файлове. Приложения.}
% Файлове и потоци от данни. Видове файлове и основни операции с файлове. Четене и форматирано писане на текстов файл с последователен достъп. Сериализиране и десериализиране на обекти. Обработка на данни от файлове. Приложения.

\chapter{Линейни структури от данни. Машинна структура от данни масив. Стек, опашка, дек. Видове линейни списъци. Реализация на всяка структура от данни като клас заедно с необходимите методи. Примери за приложения.}
% Линейни структури от данни. Машинна структура от данни масив. Стек, опашка, дек. Видове линейни списъци. Реализация на всяка структура от данни като клас заедно с необходимите методи. Примери за приложения.

\chapter{Дървовидни структури данни. Видове. Бинарно (двоично) дърво - дефиниция, типични алгоритми и програмна реализация. Обхождане на бинарно дърво за търсене -- preorder, inorder и postorder. Примери за приложения.}
% Дървовидни структури данни. Видове. Бинарно (двоично) дърво -- дефиниция, типични алгоритми и програмна реализация. Обхождане на бинарно дърво за търсене -- preorder, inorder и postorder. Примери за приложения.

\chapter{Структура от данни граф – основни понятия, дефиниция, видове, представяне и обхождане. Обхождане в дълбочина и в широчина, програмна реализация. Основни алгоритми за търсене на минимален път, за намиране на максимален подграф без цикъл и за декомпозиране на граф до свързани компоненти. Примери за приложения.}
% Структура от данни граф – основни понятия, дефиниция, видове, представяне и обхождане. Обхождане в дълбочина и в широчина, програмна реализация. Основни алгоритми за търсене на минимален път, за намиране на максимален подграф без цикъл и за декомпозиране на граф до свързани компоненти. Примери за приложения.

\chapter{Основни алгоритми за сортиране и търсене. Алгоритми за последователно и двоично търсене. Алгоритми за сортиране. Алгоритми с пълно изчерпване. Лакоми алгоритми. Метод разделяй и владей. Алгоритми с динамично програмиране. Търсене с връщане назад.}
% Основни алгоритми за сортиране и търсене. Алгоритми за последователно и двоично търсене. Алгоритми за сортиране. Алгоритми с пълно изчерпване. Лакоми алгоритми. Метод разделяй и владей. Алгоритми с динамично програмиране. Търсене с връщане назад.

