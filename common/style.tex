\theoremstyle{definition}
\newtheorem{definition}{Дефиниция}[chapter]
\newtheorem{axiom}[definition]{Аксиома}

\newtheorem{theorem}{Теорема}[chapter]
\newtheorem{lemma}[theorem]{Лема}
\newtheorem{lem}[theorem]{Твърдение}
\newtheorem{corollary}[theorem]{Следствие}

\theoremstyle{remark}
\newtheorem*{example}{Пример}

\theoremstyle{remark}
\newtheorem*{remark}{Забележка}


%\theoremstyle{plain}
%\newtheorem{exercise}{Задача}

% \renewcommand\qedsymbol{}

\renewcommand{\theoremautorefname}{Теорема}
\newcommand{\lemmaautorefname}{Лема}
\newcommand{\lemautorefname}{Твърдение}
\newcommand{\corollaryautorefname}{Следствие}

\setuptodonotes{inline}

\setcounter{tocdepth}{1}

\chapternumberfont{\large} 
\chaptertitlefont{\Large}

\usepackage[margin=2.5cm]{geometry}
