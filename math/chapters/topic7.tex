% Основни теореми за непрекъснати функции в краен и затворен интервал.

% Болцано-Вайерщрас: всяка ограничена редица има сходяща подредица

Нека $f$ -- непрекъсната в $\left[a,\,b\right]$. Тогава:

\begin{theorem}
    $f$ е ограничена в $\left[a,\,b\right]$.

    \begin{proof}
        Да допунем, че $f$ не е ограничена. По теорема на Кантор:

        \begin{enumerate}
            \item Разделяме $\left[a,\,b\right]$ на три равни части.
                  В поне един от интервалите $\left[a_1,\,b_1\right]$, $f$ не е ограничена.
            \item Разделяме $\left[a_1,\,b_1\right]$ на три равни части.
                  В поне един от интервалите $\left[a_2,\,b_2\right]$, $f$ не е ограничена.
            \item \dots
        \end{enumerate}
        Продължавайки, получаваме редица от вложени интервали $[a_i,b_i] \supset [a_{i+1},b_{i+1}], i \in \N$.
        Съществува точка $\xi \in [a_n, b_n] \forall n \in N$, тоест:

        За $\xi$, $\forall \varepsilon > 0, \exists \delta > 0: \forall x \in (\xi-\delta, \xi+\delta)$:
        \[\left|f(x) - f(\xi)\right| < \varepsilon\]
    \end{proof}
\end{theorem}