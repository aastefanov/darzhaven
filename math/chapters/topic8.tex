% Редици и редове от реални числа. Абсолютно сходящи редове.

\section{Числови редици}

\begin{definition}
    Редицата $\{a_n\}_{n=1}^{\infty}$ наричаме сходяща с граница $a$, ако:

    \[\forall\varepsilon>0\,\exists n_0 : \forall n > n_0,\, \abs{a_n-a} < \varepsilon\]

    Означаваме $\lim\limits_{n\to\infty}a_n = a$ или за краткост $\{a_n\}_{n=1}^{\infty} \to \infty$.
\end{definition}

\begin{theorem}
    Сходящите числови редици имат следните свойства:
    \begin{itemize}
        \item $\lim\limits_{n\to\infty}a_n \pm \lim\limits_{n\to\infty}b_n = \lim\limits_{n\to\infty}\left(a_n\pm b_n\right)$
        \item $\lim\limits_{n\to\infty}a_n\lim\limits_{n\to\infty}b_n = \lim\limits_{n\to\infty}\left(a_n b_n\right)$
        \item $\dfrac{\lim\limits_{n\to\infty}a_n}{\lim\limits_{n\to\infty}b_n} = \lim\limits_{n\to\infty}\left(\dfrac{a_n}{b_n}\right)$, ако $\lim\limits_{n\to\infty}b_n \neq 0$
        \item $\lim\limits_{n\to\infty}a_n \leq \lim\limits_{n\to\infty}b_n$, ако $\forall n,\, a_n \leq b_n$
    \end{itemize}

    \begin{proof}
        Нека $a_n \to a$ и $b_n \to b$. Нека $\varepsilon > 0$. Тогава, по дефиниция:

        \[\left|\begin{matrix}
                \exists & n_1 : & \forall n > n_1, & \abs{a_n-a} & < \varepsilon \\
                \exists & n_2 : & \forall n > n_2, & \abs{b_n-b} & < \varepsilon
            \end{matrix}\right.\]

        Разглеждаме $n > \max\left\{n_1,n_2\right\}$.

        \begin{enumerate}
            \item Събираме неравенствата:
                  \begin{alignat*}{3}
                      a-\varepsilon & < a_n & < a+\varepsilon \\
                      b-\varepsilon & < b_n & < b+\varepsilon
                  \end{alignat*}

                  \[(a+b)-2\varepsilon < a_n + b_n < (a+b)+2\varepsilon\]

                  Тоест $\forall\varepsilon_1 = \frac{\varepsilon}{2} > 0,\, n > d\max{n_1,n_2}, \abs{a_n+b_n-(a+b)} < \varepsilon_1$.

                  Достигнахме $\lim\limits_{n\to\infty}a_n+\lim\limits_{n\to\infty}b_n = a+b = \lim\limits_{n\to\infty}(a_n+b_n)$.

            \item Събираме неравенствата:
                  \begin{alignat*}{3}
                      -\varepsilon & < a_n - a & < \varepsilon \\
                      -\varepsilon & < b - b_n & < \varepsilon
                  \end{alignat*}

                  \[\abs{a_n - b_n - (a-b)} < 2\varepsilon\]

                  Получаваме $\lim\limits_{n\to\infty}a_n-\lim\limits_{n\to\infty}b_n = a-b = \lim\limits_{n\to\infty}(a_n-b_n)$.

        \end{enumerate}
            \item \todo{умножение}
            \item \todo{деление}
            \item \todo{неравенство}
    \end{proof}
\end{theorem}

\begin{theorem}
    Ако редицата $\left\{a_n\right\}$ е монотонно растяща и ограничена отгоре, то тя е сходяща. \\
    Аналогично, редицата е сходяща, когато е монотонно намаляваща и ограничена отдолу.
    
    \begin{proof}
        Нека редицата $\left\{a_n\right\}$ е монотонно растяща и ограничена отгоре.
        Тогава $a_{n+1} > a_n$ и $a_n < M$, където $M$ е горната граница на редицата.

        Да допуснем, че редицата не е сходяща. Тогава $\exists \varepsilon > 0 : $
        \todo{недовършено}
    \end{proof}
\end{theorem}

\begin{definition}
    Числов ред наричаме сумата $\sum\limits_{i=1}^\infty a_i = \lim\limits_{n\to\infty}\sum\limits_{i=0}^na_i$. \\
    Редът е схдоящ, ако редицата $\sum\limits_{i=0}^na_i$ е сходяща.
\end{definition}