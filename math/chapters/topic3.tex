% Комплексни числа. Алгебричен и тригонометричен вид на комплексно число. Формули на Моавър и n-ти корени на единицата.
\section{Копмлексни числа}

\begin{definition}[Множество на комплексните числа]
    \[\Cpx = \left\{ x + iy \mid x,\,y \in \R,\, i^2 = -1 \right\}\]

    Елементите на $\Cpx$ наричаме комплексни числа.
\end{definition}

\begin{theorem}
    Нека $x = a+ib,\, y = c+id,\, z = f+ig \in \Cpx$. Тогава:
    \begin{enumerate}
        \item $x + y = y + x = (a+c) + i(b+d)$
        \item $xy = yx = (ac - bd) + i(ad + bc)$
        \item $x(y + z) = xy + xz$
    \end{enumerate}

    \todo{Третото условие да се доразпише}
    \begin{proof}
        Нека $x, y, z \in \Cpx$. Тогава:
        \begin{enumerate}
            \item $x + y = a + ib + c + id = (a + c) + i(b + d)$
            \item $xy = (a+ic)(b+id) = ab + iad + ibc + i^2cd = (ab - cd) + i(ad + bc)$
            \item $x(y + z) = (a + ib)(c + f + id + ig) = ...$
        \end{enumerate}
    \end{proof}
\end{theorem}

\begin{definition}[Алгебричен и тригонометричен вид на комплексно число]
    Числото $x \in \Cpx$ е записано в тригонометричен вид, ако има вида:
    \[x = r(\cos\varphi + i\sin\varphi),\, r \in \R^, r \geq 0, \varphi \in \left[0,2\pi\right)\]
\end{definition}

\begin{lem}
    Нека $x = a+ib$. Тогава то може да се запише в тригонометричен вид.
    \begin{proof}
        \begin{align*}
            x &= a + ib \\
            x &= 
        \end{align*}
    \end{proof}
    \todo{Разписано от учебника}
\end{lem}

\section{Формули на Моавър. Корени на единицата}

\begin{definition}[Формули на Моавър]
    Нека $z = r(\cos\varphi + i\sin\varphi) \in \Cpx$. Тогава са в сила:
    \begin{itemize}
        \item Формула на Моавър за степенуване:
            \[z^n = r^n\left(\sin{nx} + i\cos{nx}\right)\]
        
        \item Формула на Моавър за коренуване:
            \[z^{-n} = \sqrt[n]{r}\left(\cos\dfrac{x+k\pi}{n} + i\sin\dfrac{x+k\pi}{n}\right),\, k = \overline{0,n-1}\]
    \end{itemize}
\end{definition}

\begin{definition}[$n$-ти корени на единицата]
    Числата $z \in \Cpx$, за които $z^n = 1$ се наричат $n$-ти корени на единицата.
\end{definition}

\begin{corollary}
    $n$-тите корени на единицата са точно $n$ на брой и имат вида
    \[\omega_k = \cos\frac{k\pi}{n} + i\sin\frac{k\pi}{n},\, k = \overline{0,n-1}\]
    \begin{proof}
        \todo{Трябва ли да се доказва?}
%        Директно следствие от  формула на Моавър:
    \end{proof}
\end{corollary}

\begin{corollary}
    Ако $\omega_k$ е $k$-ти корен на единицата, то $\omega_k = \omega_1^k$.
    \begin{proof}
        Директно следствие от втората формула на Моавър:
        \[\omega_1^k = \cos\frac{k\pi}{n} + i\sin\frac{k\pi}{n} = \omega_k\] 
    \end{proof}
\end{corollary}