% Базис, размерност, координати. Системи линейни уравнения. Теорема на Руше. Връзка между решенията на хомогенна и нехомогенна система линейни уравнения.
\section{Базис, размерност, координати}

Нека $V$ е линейно пространство над поле $F$.

Тогава:
\begin{definition}[Базис]
    Векторите $v_1,\dots,v_n \in V$ образуват базис на $V$, ако са линейно независими и всичките им линейни комбинации пораждат $V$, т.е.
    \[V = \ell(v_1,\dots,v_n) \left\{\sum_{i=1}^n \alpha_iv_i \mid \alpha_i \in F\right\}\]
\end{definition}

\begin{theorem}
    Ако $V$ е крайномерно, то всеки два негови базиса са съставени от равен брой вектори.
    \begin{proof}
        \todo{Валидно ли е?}
        Да допуснем, че $v_1,\dots,v_n$ и $w_1,\dots,w_m$ са два различни базиса на $V$, като Б.О.О. $n < m$.

        Всеки от векторите $w_k$ можем да представим във вида $w_k = \displaystyle\sum_{i=1}^n\alpha_{ik}v_i$.
        Също така, $v_l = \displaystyle\sum_{j=1}^m\beta_{jl}w_j$

        Тогава $w_{m} = \displaystyle\sum_{i=1}^n\alpha_{im}v_i = \displaystyle\sum_{j=1}^{m-1}\beta_{jl}w_j$, т.е. $w_k$ са линейно зависими -- противоречие.

        Достигнахме $n = m$.
    \end{proof}
\end{theorem}

\begin{definition}[Размерност]
    Броят на векторите $n$ във всеки базис на $V$ наричаме размерност на $V$. Записваме $\dim V = n$.
\end{definition}

\begin{theorem}
    Нека $v_1,\dots,v_n$ е базис на $V$. Тогава всеки вектор $v \in V$ се записва по единствен начин като линейна комбинация на $v_1,\dots,v_n$.
    \begin{proof}
        Да допуснем, че $v = \displaystyle\sum_{i=1}^n\alpha_iv_i = \displaystyle\sum_{i=1}^n\beta_iv_i$.
        
        Тогава $o = v - v = \displaystyle\sum_{i=1}^n(\alpha_i - \beta_i)v_i$, т.е. $\alpha_i = \beta_i$.
    \end{proof}
\end{theorem}

\begin{definition}[Координати на вектор относно базис]
    Нека $V$ -- линейно пространство и $v_1,\dots,v_n$ е негов базис.
    
    Координати на вектора $v = \alpha_1v_1+\cdots+\alpha_nv_n$ в базиса $v_1,\dots,v_n$ наричаме числата $\alpha_1,\dots,\alpha_n$.
\end{definition}

\begin{theorem}
    Линейното пространство $V$ има размерност $n$ тогава и само тогава, когато във $V$ съществуват $n$ на брой линейно независими вектора и всеки $n+1$ вектора са линейно зависими.
    \begin{proof}
        Нека $\dim V = n$. Тогава съществува базис $v_1,\dots,v_n$.

        Нека $w_1,\dots,w_{n+1} \in V$. Тогава $w_1,\dots,w_{n+1} \in V = \ell(v_1,\dots,v_n)$, т.е. са линейно зависими.
    \end{proof}
    \begin{proof}
        Нека $v_1,\dots,v_n$ са линейно независими. Да допуснем, че не образуват базис на $V$, т.е. $\ell(v_1,\dots,v_n) < V$.
        Тогава съществува вектор $v \in V \setminus \ell(v_1,\dots,v_n)$, но той е линейно независим с $v_1,\dots,v_n$. Противорчеие.
    \end{proof}
\end{theorem}

\section{Системи линейни уравнения}
Можем да разгледаме следната система линейни уравнения:

\[\left|\begin{matrix}
a_{11}x_1+&a_{12}x_2+&+\cdots+&a_{1n}x_n = b_1 \\
a_{21}x_1+&a_{22}x_2+&+\cdots+&a_{2n}x_n = b_1 \\
\vdots \\
a_{m1}x_1+&a_{m2}x_2+&+\cdots+&a_{mn}x_n = b_m
\end{matrix}\right.\]

Горната система е еквивалентна на уравнението:
\[
\begin{pmatrix}
a_{11} & a_{12} & \cdots & {a_1n} \\
a_{21} & a_{22} & \cdots & {a_2n} \\
\vdots \\
a_{m1} & a_{m2} & \cdots & {a_mn}
\end{pmatrix}
\begin{pmatrix}
    x_1 \\ x_2 \\ \vdots \\ x_n
\end{pmatrix}
=
\begin{pmatrix}
    b_1 \\ b_2 \\ \vdots \\ b_m
\end{pmatrix}
\]