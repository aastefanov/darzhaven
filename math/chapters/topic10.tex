% Определен интеграл. Дефиниция и свойства. Интегруемост на непрекъснатите функции. Теорема на Нютон-Лайбниц.

Можем да дефинираме определения интеграл по два начина -- чрез суми на Дарбу или суми на Риман.

\section{Дефиниране на определения интеграл}

\subsection{Дефиниция -- чрез суми на Дарбу}
\begin{definition}[Суми на Дарбу]
    Нека $f: [a,b] \to \R$ -- ограничена и $\left\{x_i\right\}_0^{n}$ -- разбиване на $[a,b]$.

    За всяко разбиване, нека:

    \[m_i = \sup_{x \in [x_i,x_{i+1}]} f(x),\, M_i = \inf_{x \in [x_i,x_{i+1}]} f(x),\,\Delta{x_i} = x_{i+1} - x_i\]

    Тогава дефинираме малка и голяма сума на Дарбу съответно като:
    \[s = \sum_{i=0}^n M_i \Delta x_i\]
    \[S = \sum_{i=0}^n m_i\Delta x_i\]
\end{definition}

\begin{theorem}
    При добавяне на нови точки в разбиването, голямата сума на Дарбу намалява, а малката нараства.

    \begin{proof}
        Да добавим точка $x'$ между $x_{i_0}$ и $x_{i_0+1}$.

        Знаем, че $f$ e ограничена, тогава:
        \begin{align*}
            m_1' & = \inf_{x\in[x_{i_0},x']} f(x)   \geq m_{i_0}   &
            m_2' & = \inf_{x\in[x_{i_0},x']} f(x)   \geq m_{i_0+1}   \\
            M_1' & = \sup_{x\in[x',x_{i_0}+1]} f(x) \leq M_{i_0}   &
            M_2' & = \sup_{x\in[x',x_{i_0}+1]} f(x) \leq M_{i_0+1}
        \end{align*}

        За това разбиване сумите на Дарбу са:
        \begin{alignat*}{3}
            s' & = \sum_{i\neq i_0}m_i\Delta x_i + m_1'(x'-x_{i_0}) + m_2' (x_{i_0+1}-x') & \geq s \\
            S' & = \sum_{i\neq i_0}M_i\Delta x_i + M_1'(x'-x_{i_0}) + M_2' (x_{i_0+1}-x') & \leq S
        \end{alignat*}

        Получихме $s \leq s' \leq S' \leq S$.
    \end{proof}
\end{theorem}

\begin{definition}[Долен и горен интеграл на Дарбу]
    Долен и горен интеграл на Дарбу дефинираме съответно като:
    \begin{align*}
        \underline{I} & = \lim_{\Delta{x} \to 0} s \\
        \overline{I}  & = \lim_{\Delta{x} \to 0} S
    \end{align*}

\end{definition}

\begin{definition}[Определен интеграл в смисъла на Дарбу]
    $f: [a,b] \to \R$ е интегруема в смисъл на Дарбу, ако $\underline{I} = \overline{I}$.

    Означаваме:
    \[\underline{I} = \overline{I} = I = \int_{a}^b f(x)\,dx\]
\end{definition}

\begin{theorem}[НДУ за интегруемост в смисъл на Дарбу]
    Функцията $f: [a,b] \to \R$ е интегруема в смисъл на Дарбу тогава и само тогава, когато
    \[\forall \varepsilon > 0 \exists\left\{[x,x']\right\} \subset [a,b] : S-s < \varepsilon\]

    \begin{proof}
        Нека $f$ е интегруема. Тогава $\underline{I} = \overline{I}$, т.е. $\lim_{\Delta{x} \to 0} s =  \lim_{\Delta{x} \to 0} S$.
    \end{proof}
    \begin{proof}
        Нека $\varepsilon > 0$ и за разбиването $S-s<\varepsilon$.

        Знаем че $s \leq \underline{I} \leq \overline{I} \leq S$, тогава $\underline{I} - \overline{I} < \varepsilon$.
        По дефиниция,
        \[\lim_{\Delta x \to 0} \underline{I} = \overline{I}\]

        Тогава $f$ -- интегруема.
    \end{proof}
\end{theorem}

\subsection{Дефиниция -- чрез суми на Риман}
Можем да разгледаме следната дефиниция -- на Риман:

\begin{definition}
    Нека $f: [a,b] \to \R$ -- ограничена и $\left\{x_i\right\}_0^{n}$ -- разбиване на $[a,b]$.

    Дефинираме:
    \[\xi_i = \frac{x_{i+1}+x_i}{2},\,\Delta{x_i} = x_{i+1}-x_i\]

    Сума на Риман наричаме сумата:
    \[\sigma = \sum_{i=0}^n f(\xi_i)\Delta{x_i}\]

    Определен интеграл (в смисъла на Риман) дефинираме като:
    \[\int_a^b f(x)\,dx = \lim_{\Delta x_i \to 0} \sigma\]
\end{definition}

\subsection{Връзка между дефинициите}

\begin{theorem}\label{int:darburiman}
    Ако една функция е интегруема в смисъла на Дарбу, то тя е интегруема и в смисъла на Риман.

    \begin{proof}
        Нека $f$ -- интегруема и $\int\limits_a^b f(x)\,dx = I$. Нека $\{x_i\}$ -- разбиване.

        Тогава $\lim\limits_{\Delta{x_i}\to0} s = I = \lim\limits_{\Delta{x_i}\to0} S$.

        \begin{alignat*}{3}
            s        & = \sum m_i \Delta{x_i}      & ,\,\, m_i = \inf_{x\in [x_i,x_{i+1}]} f(x) \\
            \sigma = & = \sum f(\xi_i) \Delta{x_i} & ,\,\, \xi_i in [x_i,x_{i+1}]               \\
            S        & = \sum M_i \Delta{x_i}      & ,\,\, M_i = \sup_{x\in [x_i,x_{i+1}]} f(x)
        \end{alignat*}

        Но $m_i \leq f(\xi) \leq M_i$, т.е. $s \leq \sigma \leq S$.
        При граничен преход:
        \[I \leq \lim_{\Delta x_i \to 0} \sigma \leq I\]

        Функцията е интегруема в смисъл на Риман, като:
        \[\int\limits_a^b f(x)\,dx = I\]
    \end{proof}
\end{theorem}

\begin{theorem}
    Функция $f$ е интегруема по Риман тогава и само тогава, когато $\forall{\varepsilon > 0},\, S-s\leq\varepsilon$.

    \begin{proof}
        Нека $\varepsilon > 0$ и $S-s<\varepsilon$. Тогава, по дефиниция, $\lim\limits_{\Delta{x}\to0}(S-s)=0$.

        Тогава $\lim S = \lim s$ и от \autoref{int:darburiman}, функцията е интегруема в смисъл на Риман.
    \end{proof}
    \begin{proof}
        \todo{обратното -- май съвсем не сме го писали}
    \end{proof}
\end{theorem}

\section{Свойства}

Определеният интеграл има следните свойства:
\begin{enumerate}
    \item е линейна функция:
          \[\int_a^b \lambda f(x) \pm \mu g(x) \,dx = \lambda \int_a^b f(x) \,dx \pm \mu \int_a^b g(x)\,dx\]
    \item е равен на нула върху интервал с дължина нула:
          \[\int_a^a f(x)\,dx = 0\]
    \item се съгласува с операцията абсолютна стойност по следния начин:
          \[\abs{\int_a^b f(x)\,dx} \leq \int_a^b \abs{f(x)}\,dx\]
\end{enumerate}

\todo{още неща има тук...}