\chapter{Уравнения на права и равнина. Формули за разстояния и ъгли.}
a) Уравнения на права. Полуравнини. Формули за разстояния и ъгли.
б) Уравнения на равнина. Полупространства. Формули за разстояния и ъгли.
\chapter{Канонични уравнения на криви от втора степен. Основни свойства на кривите от
втора степен.}
a) Окръжност. Канонично уравнение на парабола. Фокални свойства.
б) Канонични уравнения на елипса и хипербола. Фокални свойства.
\chapter{Комплексни числа. Алгебричен и тригонометричен вид на комплексно число.
Формули на Моавър и n-ти корени на единицата.}
\chapter{Базис, размерност, координати. Системи линейни уравнения. Теорема на Руше.
Връзка между решенията на хомогенна и нехомогенна система линейни
уравнения.}
\chapter{Корени на полиноми на една променлива. Принцип за сравняване на
коефициентите. Формули на Виет. Основна теорема на алгебрата. Следствия.}
\chapter{Полиноми на повече променливи. Основна теорема за симетричните полиноми.}
\chapter{Основни теореми за непрекъснати функции в краен и затворен интервал.}
\chapter{Редици и редове от реални числа. Абсолютно сходящи редове.}
\chapter{Теорема на Ферма. Теореми за средните стойности (Рол, Лагранж и Коши).
Формула на Тейлър.}
\chapter{Определен интеграл. Дефиниция и свойства. Интегруемост на непрекъснатите
функции. Теорема на Нютон-Лайбниц.}
\chapter{Случайни величини с дискретни разпределения. Биномно разпределение.}
\chapter{Сравнения. Функция на Ойлер. Теореми на Ойлер - Ферма и Уилсън.}
\chapter{Коренуване. Степен с рационален показател.}
\chapter{Забележителни неравенства.}
\chapter{Геометрични преобразувания. Еднаквости и подобности.}
\chapter{Лице на многоъгълник.}
\chapter{Теореми на Менелай и Чева. Забележителни точки в триъгълника.}
\chapter{Тристенен и многостенен ъгъл. Синусова и косинусова теореми за тристенен
ъгъл.}