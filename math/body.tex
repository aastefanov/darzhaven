\chapter{Уравнения на права и равнина. Формули за разстояния и ъгли.}
% Уравнения на права и равнина. Формули за разстояния и ъгли.
a) Уравнения на права. Полуравнини. Формули за разстояния и ъгли.
б) Уравнения на равнина. Полупространства. Формули за разстояния и ъгли.
\chapter{Канонични уравнения на криви от втора степен. Основни свойства на кривите от втора степен.}
%Канонични уравнения на криви от втора степен. Основни свойства на кривите от втора степен.

\section{Окръжност. Канонично уравнение на парабола. Фокални свойства}
\section{Канонични уравнения на елипса и хипербола. Фокални свойства}
\chapter{Комплексни числа. Алгебричен и тригонометричен вид на комплексно число. Формули на Моавър и n-ти корени на единицата.}
% Комплексни числа. Алгебричен и тригонометричен вид на комплексно число. Формули на Моавър и n-ти корени на единицата.
\section{Копмлексни числа}

\begin{definition}[Множество на комплексните числа]
    \[\Cpx = \left\{ x + iy \mid x,\,y \in \R,\, i^2 = -1 \right\}\]

    Елементите на $\Cpx$ наричаме комплексни числа.
\end{definition}

\begin{theorem}
    Нека $x = a+ib,\, y = c+id,\, z = f+ig \in \Cpx$. Тогава:
    \begin{enumerate}
        \item $x + y = y + x = (a+c) + i(b+d)$
        \item $xy = yx = (ac - bd) + i(ad + bc)$
        \item $x(y + z) = xy + xz$
    \end{enumerate}

    \todo[inline]{Третото условие да се доразпише}
    \begin{proof}
        Нека $x, y, z \in \Cpx$. Тогава:
        \begin{enumerate}
            \item $x + y = a + ib + c + id = (a + c) + i(b + d)$
            \item $xy = (a+ic)(b+id) = ab + iad + ibc + i^2cd = (ab - cd) + i(ad + bc)$
            \item $x(y + z) = (a + ib)(c + f + id + ig) = ...$
        \end{enumerate}
    \end{proof}
\end{theorem}

\begin{definition}[Алгебричен и тригонометричен вид на комплексно число]
    Числото $x \in \Cpx$ е записано в тригонометричен вид, ако има вида:
    \[x = r(\cos\varphi + i\sin\varphi),\, r \in \R^, r \geq 0, \varphi \in \left[0,2\pi\right)\]
\end{definition}

\begin{lem}
    Нека $x = a+ib$. Тогава то може да се запише в тригонометричен вид.
    \begin{proof}
        \begin{align*}
            x &= a + ib \\
            x &= 
        \end{align*}
    \end{proof}
    \todo[inline]{Разписано от учебника}
\end{lem}

\section{Формули на Моавър. Корени на единицата}

\begin{definition}[Формули на Моавър]
    Нека $z = r(\cos\varphi + i\sin\varphi) \in \Cpx$. Тогава са в сила:
    \begin{itemize}
        \item Формула на Моавър за степенуване:
            \[z^n = r^n\left(\sin{nx} + i\cos{nx}\right)\]
        
        \item Формула на Моавър за коренуване:
            \[z^{-n} = \sqrt[n]{r}\left(\cos\dfrac{x+k\pi}{n} + i\sin\dfrac{x+k\pi}{n}\right),\, k = \overline{0,n-1}\]
    \end{itemize}
\end{definition}

\begin{definition}[$n$-ти корени на единицата]
    Числата $z \in \Cpx$, за които $z^n = 1$ се наричат $n$-ти корени на единицата.
\end{definition}

\begin{corollary}
    $n$-тите корени на единицата са точно $n$ на брой и имат вида
    \[\omega_k = \cos\frac{k\pi}{n} + i\sin\frac{k\pi}{n},\, k = \overline{0,n-1}\]
    \begin{proof}
        \todo[inline]{Трябва ли да се доказва?}
%        Директно следствие от  формула на Моавър:
    \end{proof}
\end{corollary}

\begin{corollary}
    Ако $\omega_k$ е $k$-ти корен на единицата, то $\omega_k = \omega_1^k$.
    \begin{proof}
        Директно следствие от втората формула на Моавър:
        \[\omega_1^k = \cos\frac{k\pi}{n} + i\sin\frac{k\pi}{n} = \omega_k\] 
    \end{proof}
\end{corollary}
\chapter{Базис, размерност, координати. Системи линейни уравнения. Теорема на Руше. Връзка между решенията на хомогенна и нехомогенна система линейни
уравнения.}
% Базис, размерност, координати. Системи линейни уравнения. Теорема на Руше. Връзка между решенията на хомогенна и нехомогенна система линейни уравнения.
\section{Базис, размерност, координати}

Нека $V$ е линейно пространство над поле $F$.

Тогава:
\begin{definition}[Базис]
    Векторите $v_1,\dots,v_n \in V$ образуват базис на $V$, ако са линейно независими и всичките им линейни комбинации пораждат $V$, т.е.
    \[V = \ell(v_1,\dots,v_n) \left\{\sum_{i=1}^n \alpha_iv_i \mid \alpha_i \in F\right\}\]
\end{definition}

\begin{theorem}
    Ако $V$ е крайномерно, то всеки два негови базиса са съставени от равен брой вектори.
    \begin{proof}
        \todo{Валидно ли е?}
        Да допуснем, че $v_1,\dots,v_n$ и $w_1,\dots,w_m$ са два различни базиса на $V$, като Б.О.О. $n < m$.

        Всеки от векторите $w_k$ можем да представим във вида $w_k = \displaystyle\sum_{i=1}^n\alpha_{ik}v_i$.
        Също така, $v_l = \displaystyle\sum_{j=1}^m\beta_{jl}w_j$

        Тогава $w_{m} = \displaystyle\sum_{i=1}^n\alpha_{im}v_i = \displaystyle\sum_{j=1}^{m-1}\beta_{jl}w_j$, т.е. $w_k$ са линейно зависими -- противоречие.

        Достигнахме $n = m$.
    \end{proof}
\end{theorem}

\begin{definition}[Размерност]
    Броят на векторите $n$ във всеки базис на $V$ наричаме размерност на $V$. Записваме $\dim V = n$.
\end{definition}

\begin{theorem}
    Нека $v_1,\dots,v_n$ е базис на $V$. Тогава всеки вектор $v \in V$ се записва по единствен начин като линейна комбинация на $v_1,\dots,v_n$.
    \begin{proof}
        Да допуснем, че $v = \displaystyle\sum_{i=1}^n\alpha_iv_i = \displaystyle\sum_{i=1}^n\beta_iv_i$.
        
        Тогава $o = v - v = \displaystyle\sum_{i=1}^n(\alpha_i - \beta_i)v_i$, т.е. $\alpha_i = \beta_i$.
    \end{proof}
\end{theorem}

\begin{definition}[Координати на вектор относно базис]
    Нека $V$ -- линейно пространство и $v_1,\dots,v_n$ е негов базис.
    
    Координати на вектора $v = \alpha_1v_1+\cdots+\alpha_nv_n$ в базиса $v_1,\dots,v_n$ наричаме числата $\alpha_1,\dots,\alpha_n$.
\end{definition}

\begin{theorem}
    Линейното пространство $V$ има размерност $n$ тогава и само тогава, когато във $V$ съществуват $n$ на брой линейно независими вектора и всеки $n+1$ вектора са линейно зависими.
    \begin{proof}
        Нека $\dim V = n$. Тогава съществува базис $v_1,\dots,v_n$.

        Нека $w_1,\dots,w_{n+1} \in V$. Тогава $w_1,\dots,w_{n+1} \in V = \ell(v_1,\dots,v_n)$, т.е. са линейно зависими.
    \end{proof}
    \begin{proof}
        Нека $v_1,\dots,v_n$ са линейно независими. Да допуснем, че не образуват базис на $V$, т.е. $\ell(v_1,\dots,v_n) < V$.
        Тогава съществува вектор $v \in V \setminus \ell(v_1,\dots,v_n)$, но той е линейно независим с $v_1,\dots,v_n$. Противорчеие.
    \end{proof}
\end{theorem}

\section{Системи линейни уравнения}
Можем да разгледаме следната система линейни уравнения:

\[\left|\begin{matrix}
a_{11}x_1+&a_{12}x_2+&+\cdots+&a_{1n}x_n = b_1 \\
a_{21}x_1+&a_{22}x_2+&+\cdots+&a_{2n}x_n = b_1 \\
\vdots \\
a_{m1}x_1+&a_{m2}x_2+&+\cdots+&a_{mn}x_n = b_m
\end{matrix}\right.\]

Горната система е еквивалентна на уравнението:
\[
\begin{pmatrix}
a_{11} & a_{12} & \cdots & {a_1n} \\
a_{21} & a_{22} & \cdots & {a_2n} \\
\vdots \\
a_{m1} & a_{m2} & \cdots & {a_mn}
\end{pmatrix}
\begin{pmatrix}
    x_1 \\ x_2 \\ \vdots \\ x_n
\end{pmatrix}
=
\begin{pmatrix}
    b_1 \\ b_2 \\ \vdots \\ b_m
\end{pmatrix}
\]
\chapter{Корени на полиноми на една променлива. Принцип за сравняване на коефициентите. Формули на Виет. Основна теорема на алгебрата. Следствия.}
% Корени на полиноми на една променлива. Принцип за сравняване на коефициентите. Формули на Виет. Основна теорема на алгебрата. Следствия.
\chapter{Полиноми на повече променливи. Основна теорема за симетричните полиноми.}
% Полиноми на повече променливи. Основна теорема за симетричните полиноми.

\chapter{Основни теореми за непрекъснати функции в краен и затворен интервал.}
% Основни теореми за непрекъснати функции в краен и затворен интервал.

\chapter{Редици и редове от реални числа. Абсолютно сходящи редове.}
% Редици и редове от реални числа. Абсолютно сходящи редове.

\chapter{Теорема на Ферма. Теореми за средните стойности (Рол, Лагранж и Коши). Формула на Тейлър.}
% Теорема на Ферма. Теореми за средните стойности (Рол, Лагранж и Коши). Формула на Тейлър.

\section{Локални екстремуми}

\begin{definition}[Локален екстремум]
    Локални екстремуми са локалните минимуми и максимуми:

    Функцията $f$ има локален максимум в точка $\xi$, ако
    $\exists \varepsilon > 0:\, \forall x \in(\xi-\varepsilon,\xi+\varepsilon),\,f(x) \leq f(\xi)$.

    Функцията $f$ има локален минимум в точка $\xi$, ако
    $\exists \varepsilon > 0:\, \forall x \in(\xi-\varepsilon,\xi+\varepsilon),\,f(x) \geq f(\xi)$.
\end{definition}

\begin{theorem}[Необходимо условие за локален екстремум]
    Ако $f$ има локален ектремум\\ в точка $\xi$, то или $f$ не е диференцируема в $\xi$, или $f'(x)=0$.

    \begin{proof}
        Ако не е диференцируема, сме готови. Нека $f$ -- диференцируема в $\xi$. \\
        Ще разгледаме случая, в който $f(\xi)$ -- локален максимум.

        \begin{alignat*}{3}
            f_{-}'(\xi) &= \lim_{x\to\xi^-}\frac{f(x)-f(\xi)}{x-\xi} &\geq 0 \\
            f_{+}'(\xi) &= \lim_{x\to\xi^+}\frac{f(x)-f(\xi)}{x-\xi} &\leq 0
        \end{alignat*}

        Тогава $f'(\xi) = 0$.
    \end{proof}
\end{theorem}

\section{Теореми}
Нека функцията $f$ е непрекъсната в $[a,b]$ и притежава производна в $(a,b)$.

\begin{theorem}[Теорема на Рол]\label{extr:rolle}
    Ако $f(a) = f(b)$, то съществува $c \in (a,b)$, такова че $f'(c) = 0$.

    \begin{proof}
        Ако $f = \operatorname{const}$, то $f'(c) = 0,\,\forall c \in (a,b)$.

        Разглеждаме $f \neq \operatorname{const}$.
        Знаем, че $f$ достига своите минимум и максимум (по Вайерщрас).

        Нека $A = \min{f(x)} = f(x_A) < B = \max{f(x)} = f(x_B)$.

        Ако $A \neq f(a)$, то $f(x) \geq f(x_A) \forall x \in (a,b)$ и тогава $f'(x_A) = 0$.

        Ако $B \neq f(a)$, то $f(x) \leq f(x_B) \forall x \in (a,b)$ и тогава $f'(x_B) = 0$.

        Тъй като $A < B$, намерихме поне един локален екстремум, в който производната е равна на нула.
    \end{proof}
\end{theorem}

\begin{theorem}[Теорема на Лагранж]\label{extr:lagrange}
    Съществува $c \in (a,b)$, такова че $f(b)-f(a) = f'(c)(b-a)$.

    \begin{proof}
   %     Ако $f(a) = f(b)$, разглеждаме \autoref{extr:rolle}.
    Това, което търсим, е аналогично на $f'(c) = \dfrac{f(b)-f(a)}{b-a}$.

    Нека $g(x) = f(x) - \dfrac{f(b)-f(a)}{b-a}(x-a)$.

    Тогава:

    \begin{alignat*}{4}
        g(a) &= f(a) -  \dfrac{f(b)-f(a)}{b-a}(a-a) &= f(a) - 0 &= f(a) \\
        g(b) &= f(b) - \dfrac{f(b)-f(a)}{b-a}(b-a) &= f(b) - f(b) + f(a) &= f(a)
    \end{alignat*}

    Прилагаме \autoref{extr:rolle} за $g$. Съществува $\xi \in (a,\,b): g'(\xi) = 0$.

    \[0 = g'(\xi) = f'(\xi) - \dfrac{f(b)-f(a)}{b-a}\]
    \[f'(\xi) = \dfrac{f(b)-f(a)}{b-a}\]
    \end{proof}
\end{theorem}

\begin{theorem}[Теорема на Коши -- обобщен вид на теоремата на Лагранж]
    Ако $g : [a,b]\to\R$ -- непрекъсната, диференцируема в (a,b), то съществува $c \in (a,b)$, такова че:
    \[\frac{f(b)-f(a)}{g(b)-g(a)} = \frac{f'(c)}{g'(c)}\]

    \begin{proof}
        Разглеждаме $g(a) \neq g(b)$, в противен случай уравнението няма смисъл.

        \todo{довършване}
    \end{proof}
\end{theorem}

\section{Формула на Тейлър}
\chapter{Определен интеграл. Дефиниция и свойства. Интегруемост на непрекъснатите функции. Теорема на Нютон-Лайбниц.}
% Определен интеграл. Дефиниция и свойства. Интегруемост на непрекъснатите функции. Теорема на Нютон-Лайбниц.

Можем да дефинираме определения интеграл по два начина -- чрез суми на Дарбу или суми на Риман.

\section{Дефиниране на определения интеграл}

\subsection{Дефиниция -- чрез суми на Дарбу}
\begin{definition}[Суми на Дарбу]
    Нека $f: [a,b] \to \R$ -- ограничена и $\left\{x_i\right\}_0^{n}$ -- разбиване на $[a,b]$.

    За всяко разбиване, нека:

    \[m_i = \sup_{x \in [x_i,x_{i+1}]} f(x),\, M_i = \inf_{x \in [x_i,x_{i+1}]} f(x),\,\Delta{x_i} = x_{i+1} - x_i\]

    Тогава дефинираме малка и голяма сума на Дарбу съответно като:
    \[s = \sum_{i=0}^n M_i \Delta x_i\]
    \[S = \sum_{i=0}^n m_i\Delta x_i\]
\end{definition}

\begin{theorem}
    При добавяне на нови точки в разбиването, голямата сума на Дарбу намалява, а малката нараства.

    \begin{proof}
        Да добавим точка $x'$ между $x_{i_0}$ и $x_{i_0+1}$.

        Знаем, че $f$ e ограничена, тогава:
        \begin{align*}
            m_1' & = \inf_{x\in[x_{i_0},x']} f(x)   \geq m_{i_0}   &
            m_2' & = \inf_{x\in[x_{i_0},x']} f(x)   \geq m_{i_0+1}   \\
            M_1' & = \sup_{x\in[x',x_{i_0}+1]} f(x) \leq M_{i_0}   &
            M_2' & = \sup_{x\in[x',x_{i_0}+1]} f(x) \leq M_{i_0+1}
        \end{align*}

        За това разбиване сумите на Дарбу са:
        \begin{alignat*}{3}
            s' & = \sum_{i\neq i_0}m_i\Delta x_i + m_1'(x'-x_{i_0}) + m_2' (x_{i_0+1}-x') & \geq s
            S' & = \sum_{i\neq i_0}M_i\Delta x_i + M_1'(x'-x_{i_0}) + M_2' (x_{i_0+1}-x') & \leq S
        \end{alignat*}

        Получихме $s \leq s' \leq S' \leq S$.
    \end{proof}
\end{theorem}

\begin{definition}[Долен и горен интеграл на Дарбу]
    Долен и горен интеграл на Дарбу дефинираме съответно като:
    \begin{align*}
        \underline{I} & = \lim_{\Delta{x} \to 0} s \\
        \overline{I}  & = \lim_{\Delta{x} \to 0} S
    \end{align*}

\end{definition}

\begin{definition}[Определен интеграл в смисъла на Дарбу]
    $f: [a,b] \to \R$ е интегруема в смисъл на Дарбу, ако $\underline{I} = \overline{I}$.

    Означаваме:
    \[\underline{I} = \overline{I} = I = \int_{a}^b f(x)\,dx\]
\end{definition}

\begin{theorem}[НДУ за интегруемост в смисъл на Дарбу]
    Функцията $f: [a,b] \to \R$ е интегруема в смисъл на Дарбу тогава и само тогава, когато
    \[\forall \varepsilon > 0 \exists\left\{[x,x']\right\} \subset [a,b] : S-s < \varepsilon\]

    \begin{proof}
        Нека $f$ е интегруема. Тогава $\underline{I} = \overline{I}$, т.е. $\lim_{\Delta{x} \to 0} s =  \lim_{\Delta{x} \to 0} S$.
    \end{proof}
    \begin{proof}
        Нека $\varepsilon > 0$ и за разбиването $S-s<\varepsilon$.
        
        Знаем че $s \leq \underline{I} \leq \overline{I} \leq S$, тогава $\underline{I} - \overline{I} < \varepsilon}$.        По дефиниция,
        \[\lim_{\Delta x \to 0} \underline{I} = \overline{I}\]

        Тогава $f$ -- интегруема.
    \end{proof}
\end{theorem}

\subsection{Дефиниция -- чрез суми на Риман}
Можем да разгледаме следната дефиниция -- на Риман:

\begin{definition}
    Нека $f: [a,b] \to \R$ -- ограничена и $\left\{x_i\right\}_0^{n}$ -- разбиване на $[a,b]$.

    Дефинираме:
    \[\xi_i = \frac{x_{i+1}+x_i}{2},\,\Delta{x_i} = x_{i+1}-x_i\]

    Сума на Риман наричаме сумата:
    \[\sigma = \sum_{i=0}^n f(\xi_i)\Delta{x_i}\]

    Определен интеграл (в смисъла на Риман) дефинираме като:
    \[\int_a^b f(x)\,dx = \lim_{\Delta x_i \to 0} \sigma\]
\end{definition}

\subsection{Еквивалентност на дефинициите}

\section{Свойства}

Определеният интеграл има следните свойства:
\begin{theorem}
    \begin{proof}

    \end{proof}
\end{theorem}
\chapter{Случайни величини с дискретни разпределения. Биномно разпределение.}
% Случайни величини с дискретни разпределения. Биномно разпределение.

\chapter{Сравнения. Функция на Ойлер. Теореми на Ойлер-Ферма и Уилсън.}
% Сравнения. Функция на Ойлер. Теореми на Ойлер-Ферма и Уилсън.

\chapter{Коренуване. Степен с рационален показател.}
% Коренуване. Степен с рационален показател.

\chapter{Забележителни неравенства.}
% Забележителни неравенства.

\chapter{Геометрични преобразувания. Еднаквости и подобности.}
% Геометрични преобразувания. Еднаквости и подобности.

\chapter{Лице на многоъгълник.}
% Лице на многоъгълник.

\chapter{Теореми на Менелай и Чева. Забележителни точки в триъгълника.}
% Теореми на Менелай и Чева. Забележителни точки в триъгълника.

\chapter{Тристенен и многостенен ъгъл. Синусова и косинусова теореми за тристенен ъгъл.}
% Тристенен и многостенен ъгъл. Синусова и косинусова теореми за тристенен ъгъл.
