\chapter{Уравнения на права и равнина. Формули за разстояния и ъгли.}
% Уравнения на права и равнина. Формули за разстояния и ъгли.
\section{Уравнения на права. Полуравнини. Формули за разстояния и ъгли}
\section{Уравнения на равнина. Полупространства. Формули за разстояния и ъгли}
\chapter{Канонични уравнения на криви от втора степен. Основни свойства на кривите от втора степен.}
%Канонични уравнения на криви от втора степен. Основни свойства на кривите от втора степен.

а) Окръжност. Канонично уравнение на парабола. Фокални свойства.
б) Канонични уравнения на елипса и хипербола. Фокални свойства.
\chapter{Комплексни числа. Алгебричен и тригонометричен вид на комплексно число. Формули на Моавър и n-ти корени на единицата.}
% Комплексни числа. Алгебричен и тригонометричен вид на комплексно число. Формули на Моавър и n-ти корени на единицата.
\chapter{Базис, размерност, координати. Системи линейни уравнения. Теорема на Руше. Връзка между решенията на хомогенна и нехомогенна система линейни
уравнения.}
% Базис, размерност, координати. Системи линейни уравнения. Теорема на Руше. Връзка между решенията на хомогенна и нехомогенна система линейни уравнения.
\section{Базис, размерност, координати}

Нека $V$ е линейно пространство над поле $F$.

Тогава:
\begin{definition}[Базис]
    Векторите $v_1,\dots,v_n \in V$ образуват базис на $V$, ако са линейно независими и всичките им линейни комбинации пораждат $V$, т.е.
    \[V = \ell(v_1,\dots,v_n) \left\{\sum_{i=1}^n \alpha_iv_i \mid \alpha_i \in F\right\}\]
\end{definition}

\begin{theorem}
    Ако $V$ е крайномерно, то всеки два негови базиса са съставени от равен брой вектори.
    \begin{proof}
        \todo[inline]{Валидно ли е?}
        Да допуснем, че $v_1,\dots,v_n$ и $w_1,\dots,w_m$ са два различни базиса на $V$, като Б.О.О. $n < m$.

        Всеки от векторите $w_k$ можем да представим във вида $w_k = \displaystyle\sum_{i=1}^n\alpha_{ik}v_i$.
        Също така, $v_l = \displaystyle\sum_{j=1}^m\beta_{jl}w_j$

        Тогава $w_{m} = \displaystyle\sum_{i=1}^n\alpha_{im}v_i = \displaystyle\sum_{j=1}^{m-1}\beta_{jl}w_j$, т.е. $w_k$ са линейно зависими -- противоречие.

        Достигнахме $n = m$.
    \end{proof}
\end{theorem}

\begin{definition}[Размерност]
    Броят на векторите $n$ във всеки базис на $V$ наричаме размерност на $V$. Записваме $\dim V = n$.
\end{definition}

\begin{theorem}
    Нека $v_1,\dots,v_n$ е базис на $V$. Тогава всеки вектор $v \in V$ се записва по единствен начин като линейна комбинация на $v_1,\dots,v_n$.
    \begin{proof}
        Да допуснем, че $v = \displaystyle\sum_{i=1}^n\alpha_iv_i = \displaystyle\sum_{i=1}^n\beta_iv_i$.
        
        Тогава $o = v - v = \displaystyle\sum_{i=1}^n(\alpha_i - \beta_i)v_i$, т.е. $\alpha_i = \beta_i$.
    \end{proof}
\end{theorem}

\begin{definition}[Координати на вектор относно базис]
    Нека $V$ -- линейно пространство и $v_1,\dots,v_n$ е негов базис.
    
    Координати на вектора $v = \alpha_1v_1+\cdots+\alpha_nv_n$ в базиса $v_1,\dots,v_n$ наричаме числата $\alpha_1,\dots,\alpha_n$.
\end{definition}

\begin{theorem}
    Линейното пространство $V$ има размерност $n$ тогава и само тогава, когато във $V$ съществуват $n$ на брой линейно независими вектора и всеки $n+1$ вектора са линейно зависими.
    \begin{proof}
        Нека $\dim V = n$. Тогава съществува базис $v_1,\dots,v_n$.

        Нека $w_1,\dots,w_{n+1} \in V$. Тогава $w_1,\dots,w_{n+1} \in V = \ell(v_1,\dots,v_n)$, т.е. са линейно зависими.
    \end{proof}
    \begin{proof}
        Нека $v_1,\dots,v_n$ са линейно независими. Да допуснем, че не образуват базис на $V$, т.е. $\ell(v_1,\dots,v_n) < V$.
        Тогава съществува вектор $v \in V \setminus \ell(v_1,\dots,v_n)$, но той е линейно независим с $v_1,\dots,v_n$. Противорчеие.
    \end{proof}
\end{theorem}

\section{Системи линейни уравнения}
\chapter{Корени на полиноми на една променлива. Принцип за сравняване на коефициентите. Формули на Виет. Основна теорема на алгебрата. Следствия.}
% Корени на полиноми на една променлива. Принцип за сравняване на коефициентите. Формули на Виет. Основна теорема на алгебрата. Следствия.
\chapter{Полиноми на повече променливи. Основна теорема за симетричните полиноми.}
% Полиноми на повече променливи. Основна теорема за симетричните полиноми.

\chapter{Основни теореми за непрекъснати функции в краен и затворен интервал.}
% Основни теореми за непрекъснати функции в краен и затворен интервал.

% Болцано-Вайерщрас: всяка ограничена редица има сходяща подредица

Нека $f$ -- непрекъсната в $\left[a,\,b\right]$. Тогава:

\begin{theorem}
    $f$ е ограничена в $\left[a,\,b\right]$.

    \begin{proof}
        Да допунем, че $f$ не е ограничена. По теорема на Кантор:

        \begin{enumerate}
            \item Разделяме $\left[a,\,b\right]$ на три равни части.
                  В поне един от интервалите $\left[a_1,\,b_1\right]$, $f$ не е ограничена.
            \item Разделяме $\left[a_1,\,b_1\right]$ на три равни части.
                  В поне един от интервалите $\left[a_2,\,b_2\right]$, $f$ не е ограничена.
            \item \dots
        \end{enumerate}
        Продължавайки, получаваме редица от вложени интервали $[a_i,b_i] \supset [a_{i+1},b_{i+1}], i \in \N$.
        Съществува точка $\xi \in [a_n, b_n] \forall n \in N$, тоест:

        За $\xi$, $\forall \varepsilon > 0, \exists \delta > 0: \forall x \in (\xi-\delta, \xi+\delta)$:
        \[\left|f(x) - f(\xi)\right| < \varepsilon\]
    \end{proof}
\end{theorem}
\chapter{Редици и редове от реални числа. Абсолютно сходящи редове.}
% Редици и редове от реални числа. Абсолютно сходящи редове.

\section{Числови редици}

\begin{definition}
    Редицата $\{a_n\}_{n=1}^{\infty}$ наричаме сходяща с граница $a$, ако:

    \[\forall\varepsilon>0\,\exists n_0 : \forall n > n_0,\, \abs{a_n-a} < \varepsilon\]

    Означаваме $\lim\limits_{n\to\infty}a_n = a$ или за краткост $\{a_n\}_{n=1}^{\infty} \to \infty$.
\end{definition}

\begin{theorem}
    Сходящите числови редици имат следните свойства:
    \begin{itemize}
        \item $\lim\limits_{n\to\infty}a_n \pm \lim\limits_{n\to\infty}b_n = \lim\limits_{n\to\infty}\left(a_n\pm b_n\right)$
        \item $\lim\limits_{n\to\infty}a_n\lim\limits_{n\to\infty}b_n = \lim\limits_{n\to\infty}\left(a_n b_n\right)$
        \item $\dfrac{\lim\limits_{n\to\infty}a_n}{\lim\limits_{n\to\infty}b_n} = \lim\limits_{n\to\infty}\left(\dfrac{a_n}{b_n}\right)$, ако $\lim\limits_{n\to\infty}b_n \neq 0$
        \item $\lim\limits_{n\to\infty}a_n \leq \lim\limits_{n\to\infty}b_n$, ако $\forall n,\, a_n \leq b_n$
    \end{itemize}

    \begin{proof}
        Нека $a_n \to a$ и $b_n \to b$. Нека $\varepsilon > 0$. Тогава, по дефиниция:

        \[\left|\begin{matrix}
                \exists & n_1 : & \forall n > n_1, & \abs{a_n-a} & < \varepsilon \\
                \exists & n_2 : & \forall n > n_2, & \abs{b_n-b} & < \varepsilon
            \end{matrix}\right.\]

        Разглеждаме $n > \max\left\{n_1,n_2\right\}$.

        \begin{enumerate}
            \item Събираме неравенствата:
                  \begin{alignat*}{3}
                      a-\varepsilon & < a_n & < a+\varepsilon \\
                      b-\varepsilon & < b_n & < b+\varepsilon
                  \end{alignat*}

                  \[(a+b)-2\varepsilon < a_n + b_n < (a+b)+2\varepsilon\]

                  Тоест $\forall\varepsilon_1 = \frac{\varepsilon}{2} > 0,\, n > d\max{n_1,n_2}, \abs{a_n+b_n-(a+b)} < \varepsilon_1$.

                  Достигнахме $\lim\limits_{n\to\infty}a_n+\lim\limits_{n\to\infty}b_n = a+b = \lim\limits_{n\to\infty}(a_n+b_n)$.

            \item Събираме неравенствата:
                  \begin{alignat*}{3}
                      -\varepsilon & < a_n - a & < \varepsilon \\
                      -\varepsilon & < b - b_n & < \varepsilon
                  \end{alignat*}

                  \[\abs{a_n - b_n - (a-b)} < 2\varepsilon\]

                  Получаваме $\lim\limits_{n\to\infty}a_n-\lim\limits_{n\to\infty}b_n = a-b = \lim\limits_{n\to\infty}(a_n-b_n)$.

        \end{enumerate}
            \item \todo{умножение}
            \item \todo{деление}
            \item \todo{неравенство}
    \end{proof}
\end{theorem}

\begin{theorem}
    Ако редицата $\left\{a_n\right\}$ е монотонно растяща и ограничена отгоре, то тя е сходяща. \\
    Аналогично, редицата е сходяща, когато е монотонно намаляваща и ограничена отдолу.
    
    \begin{proof}
        Нека редицата $\left\{a_n\right\}$ е монотонно растяща и ограничена отгоре.
        Тогава $a_{n+1} > a_n$ и $a_n < M$, където $M$ е горната граница на редицата.

        Да допуснем, че редицата не е сходяща. Тогава $\exists \varepsilon > 0 : $
        \todo{недовършено}
    \end{proof}
\end{theorem}

\begin{definition}
    Числов ред наричаме сумата $\sum\limits_{i=1}^\infty a_i = \lim\limits_{n\to\infty}\sum\limits_{i=0}^na_i$. \\
    Редът е схдоящ, ако редицата $\sum\limits_{i=0}^na_i$ е сходяща.
\end{definition}
\chapter{Теорема на Ферма. Теореми за средните стойности (Рол, Лагранж и Коши). Формула на Тейлър.}
% Теорема на Ферма. Теореми за средните стойности (Рол, Лагранж и Коши). Формула на Тейлър.

\chapter{Определен интеграл. Дефиниция и свойства. Интегруемост на непрекъснатите функции. Теорема на Нютон-Лайбниц.}
% Определен интеграл. Дефиниция и свойства. Интегруемост на непрекъснатите функции. Теорема на Нютон-Лайбниц.

Можем да дефинираме определения интеграл по два начина -- чрез суми на Дарбу или суми на Риман.

\section{Дефиниране на определения интеграл}

\subsection{Дефиниция -- чрез суми на Дарбу}
\begin{definition}[Суми на Дарбу]
    Нека $f: [a,b] \to \R$ -- ограничена и $\left\{x_i\right\}_0^{n}$ -- разбиване на $[a,b]$.

    За всяко разбиване, нека:

    \[m_i = \sup_{x \in [x_i,x_{i+1}]} f(x),\, M_i = \inf_{x \in [x_i,x_{i+1}]} f(x),\,\Delta{x_i} = x_{i+1} - x_i\]

    Тогава дефинираме малка и голяма сума на Дарбу съответно като:
    \[s = \sum_{i=0}^n M_i \Delta x_i\]
    \[S = \sum_{i=0}^n m_i\Delta x_i\]
\end{definition}

\begin{theorem}
    При добавяне на нови точки в разбиването, голямата сума на Дарбу намалява, а малката нараства.

    \begin{proof}
        Да добавим точка $x'$ между $x_{i_0}$ и $x_{i_0+1}$.

        Знаем, че $f$ e ограничена, тогава:
        \begin{align*}
            m_1' & = \inf_{x\in[x_{i_0},x']} f(x)   \geq m_{i_0}   &
            m_2' & = \inf_{x\in[x_{i_0},x']} f(x)   \geq m_{i_0+1}   \\
            M_1' & = \sup_{x\in[x',x_{i_0}+1]} f(x) \leq M_{i_0}   &
            M_2' & = \sup_{x\in[x',x_{i_0}+1]} f(x) \leq M_{i_0+1}
        \end{align*}

        За това разбиване сумите на Дарбу са:
        \begin{alignat*}{3}
            s' & = \sum_{i\neq i_0}m_i\Delta x_i + m_1'(x'-x_{i_0}) + m_2' (x_{i_0+1}-x') & \geq s
            S' & = \sum_{i\neq i_0}M_i\Delta x_i + M_1'(x'-x_{i_0}) + M_2' (x_{i_0+1}-x') & \leq S
        \end{alignat*}

        Получихме $s \leq s' \leq S' \leq S$.
    \end{proof}
\end{theorem}

\begin{definition}[Долен и горен интеграл на Дарбу]
    Долен и горен интеграл на Дарбу дефинираме съответно като:
    \begin{align*}
        \underline{I} & = \lim_{\Delta{x} \to 0} s \\
        \overline{I}  & = \lim_{\Delta{x} \to 0} S
    \end{align*}

\end{definition}

\begin{definition}[Определен интеграл в смисъла на Дарбу]
    $f: [a,b] \to \R$ е интегруема в смисъл на Дарбу, ако $\underline{I} = \overline{I}$.

    Означаваме:
    \[\underline{I} = \overline{I} = I = \int_{a}^b f(x)\,dx\]
\end{definition}

\begin{theorem}[НДУ за интегруемост в смисъл на Дарбу]
    Функцията $f: [a,b] \to \R$ е интегруема в смисъл на Дарбу тогава и само тогава, когато
    \[\forall \varepsilon > 0 \exists\left\{[x,x']\right\} \subset [a,b] : S-s < \varepsilon\]

    \begin{proof}
        Нека $f$ е интегруема. Тогава $\underline{I} = \overline{I}$, т.е. $\lim_{\Delta{x} \to 0} s =  \lim_{\Delta{x} \to 0} S$.
    \end{proof}
    \begin{proof}
        Нека $\varepsilon > 0$ и за разбиването $S-s<\varepsilon$.
        
        Знаем че $s \leq \underline{I} \leq \overline{I} \leq S$, тогава $\underline{I} - \overline{I} < \varepsilon}$.        По дефиниция,
        \[\lim_{\Delta x \to 0} \underline{I} = \overline{I}\]

        Тогава $f$ -- интегруема.
    \end{proof}
\end{theorem}

\subsection{Дефиниция -- чрез суми на Риман}
Можем да разгледаме следната дефиниция -- на Риман:

\begin{definition}
    Нека $f: [a,b] \to \R$ -- ограничена и $\left\{x_i\right\}_0^{n}$ -- разбиване на $[a,b]$.

    Дефинираме:
    \[\xi_i = \frac{x_{i+1}+x_i}{2},\,\Delta{x_i} = x_{i+1}-x_i\]

    Сума на Риман наричаме сумата:
    \[\sigma = \sum_{i=0}^n f(\xi_i)\Delta{x_i}\]

    Определен интеграл (в смисъла на Риман) дефинираме като:
    \[\int_a^b f(x)\,dx = \lim_{\Delta x_i \to 0} \sigma\]
\end{definition}

\subsection{Еквивалентност на дефинициите}

\section{Свойства}

Определеният интеграл има следните свойства:
\begin{theorem}
    \begin{proof}

    \end{proof}
\end{theorem}
\chapter{Случайни величини с дискретни разпределения. Биномно разпределение.}
% Случайни величини с дискретни разпределения. Биномно разпределение.

\chapter{Сравнения. Функция на Ойлер. Теореми на Ойлер - Ферма и Уилсън.}
% Сравнения. Функция на Ойлер. Теореми на Ойлер-Ферма и Уилсън.

\chapter{Коренуване. Степен с рационален показател.}
% Коренуване. Степен с рационален показател.

\chapter{Забележителни неравенства.}
\input{chapters/topic14}
\chapter{Геометрични преобразувания. Еднаквости и подобности.}
% Геометрични преобразувания. Еднаквости и подобности.

\chapter{Лице на многоъгълник.}
\input{chapters/topic16}
\chapter{Теореми на Менелай и Чева. Забележителни точки в триъгълника.}
% Теореми на Менелай и Чева. Забележителни точки в триъгълника.

\chapter{Тристенен и многостенен ъгъл. Синусова и косинусова теореми за тристенен ъгъл.}
% Тристенен и многостенен ъгъл. Синусова и косинусова теореми за тристенен ъгъл.
